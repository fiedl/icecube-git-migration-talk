% https://tex.stackexchange.com/q/433191/70789

\definecolor{ocre}{HTML}{800000}
\definecolor{sky}{HTML}{C6D9F1}
\definecolor{skybox}{HTML}{5F86B3}

\usepackage{tikz}
\usepackage{pgfmath}

\usetikzlibrary{decorations.text, arrows.meta,calc,shadows.blur,shadings}
\usetikzlibrary{arrows.meta, positioning}

\tikzset{
 arrow/.style = {%
    draw=red!40,
    line width=2mm,% <-- select desired width
    -{Triangle[length=3mm]},
    shorten >=1mm, shorten <=1mm,
    font=\fontsize{8}{8}\selectfont
  },
  myarrow/.style={
      thick,
      -latex,
  },
  Center/.style ={
      circle,
      fill=ocre,
      text=white,
      align=center,
      font =\\footnotesize\\bf,
      inner sep=1pt,
  },
  RedArc/.style ={
      color=black,
      thick,
      fill=ocre,
      blur shadow, %Tikzedt not suport online view
  },
  SkyArc/.style ={
      color=skybox,
      thick,
      fill=sky,
      blur shadow, %Tikzedt not suport online view
  },
}

% arctext from Andrew code with modifications:
%Variables: 1: ID, 2:Style 3:box height 4: Radious 5:start-angl 6:end-angl 7:text {format along path}
\def\arctext[#1][#2][#3](#4)(#5)(#6)#7{

\draw[#2] (#5:#4cm+#3) coordinate (above #1) arc (#5:#6:#4cm+#3)
             -- (#6:#4) coordinate (right #1) -- (#6:#4cm-#3) coordinate (below right #1) arc (#6:#5:#4cm-#3) coordinate [midway] (below #1)
             -- (#5:#4cm-#3) coordinate (below left #1) -- (#5:#4) coordinate (left #1) -- cycle;
            \def\a#1{#4cm+#3}
            \def\b#1{#4cm-#3}
\path[
    decoration={
        raise = -0.5ex, % Controls relavite text height position.
        text  along path,
        text = {#7},
        text align = center,
    },
    decorate
    ]
    (#5:#4) arc (#5:#6:#4);
}

%arcarrow, this is mine, for beerware purpose...
%Function: Draw an arrow from arctex coordinate specific nodes to another
%Arrow start at the start of arctext box and could be shifted to change the position
%to avoid go over another box.
%Var: 1:Start coordinate 2:End coordinate 3:angle to shift from acrtext box
\def\arcarrow(#1)(#2)[#3]{
    \draw[thick,->,>=latex]
        let \p1 = (#1), \p2 = (#2), % To access cartesian coordinates x, and y.
            \n1 = {veclen(\x1,\y1)}, % Distance from the origin
            \n2 = {veclen(\x2,\y2)}, % Distance from the origin
            \n3 = {atan2(\y1,\x1)} % Angle where acrtext starts.
        in (\n3-#3: \n1) -- (\n3-#3: \n2); % Draw the arrow.
}